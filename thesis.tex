\documentclass[
	11pt, 
	a4paper, 
	twoside, 
	parskip=half*, %Zeilenabstand bei Absätzen
	openright,  %openany -> auf welcher seite sollen neue Kapitel beginnen
	listof=totoc, %listings ins inhaltsverzeichnis mit aufnehmen
	bibliography=totoc, %literaturverzeichnis ins inhaltsverzeichnis aufnehmen
	index=totoc, %indexverzeichnis ins inhaltsverzeichnis aufnehmen
      toc=chapterentrywithdots, % punkte im inhaltsverzeichnis auch für kapitel setzen
      %chapterprefix=true %verändert die anzeige des kapitels, zeigt zur Kapitelnummer noch "Kapitel" an
]
{scrbook}

\usepackage[utf8]{inputenc}
\usepackage[ngerman]{babel}
\usepackage{url}
\usepackage[autostyle=true,german=quotes]{csquotes}
\usepackage[T1]{fontenc}
\usepackage{pdfpages}
\usepackage{textcomp}

% Metadaten für PDF
\usepackage[pdftex,
            pdfauthor={Jon Doe},
            pdftitle={Thesis Template},
            pdfsubject={Bachelorthesis},
            pdfkeywords={Thesis},
            pdfproducer={LaTeX},
            pdfcreator={pdfLaTeX},
            pdfduplex={DuplexFlipLongEdge}, %Alt.: Simplex oder DuplexFlipShortEdge 
            pdflang={de}, % en
]{hyperref}

% Tabellen
\usepackage{caption} 
\captionsetup[table]{belowskip=12pt,aboveskip=4pt}
\usepackage{diagbox}

% Einstellungen laden
\usepackage[
    backend=biber, 
    style=authoryear-icomp, % alphabetic
    citestyle=authoryear-icomp, % alphabetic, authortitle
    date=short,
    % backref=true, % Display pages on which the reference is used
    maxnames=1, % affects the cites
    maxbibnames=3, % affects the bibliography
    pagetracker=true,
    isbn=false,    
    % block=ragged, % break urls
    % firstinits=true, % shorten first names
    backrefstyle=three+ % Combine pages
]{biblatex}

% Distance between bibliographical references
\setlength{\bibitemsep}{.5em}

% Indentation after the first line
\setlength{\bibhang}{2em}

% URL in the bibliography is in angle brackets
\DeclareFieldFormat{url}{<\url{#1}>}

% break too long urls
\setcounter{biburllcpenalty}{7000}
\setcounter{biburlucpenalty}{8000}

% Source of the bibliography file
\bibliography{sources/literature.bib}

%grafics
\usepackage{graphicx}
\graphicspath{ {images/} }
\DeclareGraphicsExtensions{.pdf,.png,.jpg,.jpeg,.gif}
% You can visit this website to find more color codes for LaTeX
% http://latexcolor.com/s

\usepackage{xcolor}

% colors
\definecolor{white}{rgb}{1,1,1}
\definecolor{black}{rgb}{0,0,0}
\definecolor{middlegray}{rgb}{0.5,0.5,0.5}
\definecolor{lightgray}{rgb}{.95,.95,.95}
\definecolor{arsenic}{rgb}{0.23, 0.27, 0.29}
\definecolor{arsenicLight}{rgb}{0.20, 0.20, 0.20}
\definecolor{darkgray}{rgb}{.4,.4,.4}
\definecolor{purple}{rgb}{0.65, 0.12, 0.82}
\definecolor{orange}{rgb}{0.8,0.3,0.3}
\definecolor{yac}{rgb}{0.6,0.6,0.1}
\definecolor{green}{rgb}{.2,0.6,0.3}
\definecolor{azure}{rgb}{0.0, 0.5, 1.0}
\definecolor{editorGray}{rgb}{0.95, 0.95, 0.95}
\definecolor{editorOcher}{rgb}{1, 0.5, 0}
\definecolor{editorGreen}{rgb}{0, 0.5, 0}
\definecolor{orange}{rgb}{1,0.45,0.13}		
\definecolor{olive}{rgb}{0.17,0.59,0.20}
\definecolor{brown}{rgb}{0.69,0.31,0.31}
\definecolor{purple}{rgb}{0.38,0.18,0.81}
\definecolor{lightblue}{rgb}{0.1,0.57,0.7}
\definecolor{lightred}{rgb}{1,0.4,0.5}

\definecolor{vscodered}{HTML}{E53935}
\definecolor{vscodelightred}{HTML}{EF5350}
\definecolor{vscodeblue}{HTML}{1565C0}
\definecolor{vscodegreen}{HTML}{66BB6A}

\definecolor{lightblack}{HTML}{212121}
\definecolor{darkraspberry}{rgb}{0.53, 0.15, 0.34}

% blue hues
\definecolor{bleudefrance}{rgb}{0.19, 0.55, 0.91}
\definecolor{brandeisblue}{rgb}{0.0, 0.44, 1.0}
\definecolor{blue(ncs)}{rgb}{0.0, 0.53, 0.74}
\definecolor{coolblack}{rgb}{0.0, 0.18, 0.39}

% red hues
\definecolor{coralred}{rgb}{1.0, 0.25, 0.25}
\definecolor{darkred}{rgb}{0.55, 0.0, 0.0}

%geometry
\usepackage{geometry}
\geometry{left=25mm, right=25mm, top=25mm, bottom=30mm}

\usepackage[automark]{scrlayer-scrpage}
\pagestyle{scrheadings}
\automark*[section]{}
\ohead{\headmark}
\ihead{}
\ofoot{\thepage}

%footnote gap
\addtolength{\skip\footins}{1ex}
\addtolength{\footnotesep}{0.5ex}

%prevend footnote page break
\interfootnotelinepenalty=10000

% zeilenabstände
\usepackage[onehalfspacing]{setspace}

\raggedbottom %text muss nicht bis zum ende einer seite gehen

% platz vor und nach Kapitelüberschriften
\RedeclareSectionCommand[beforeskip=0pt,afterskip=.6cm,font=\fontsize{18}{20}\selectfont]{chapter}
%\RedeclareSectionCommand[beforeskip=10pt,afterskip=.3cm,font=\fontsize{18}{25}\selectfont]{section}
%\RedeclareSectionCommand[beforeskip=10pt,afterskip=.3cm,font=\fontsize{16}{25}\selectfont]{subsection}
%\RedeclareSectionCommand[beforeskip=0pt,afterskip=.3cm,font=\fontsize{14}{25}\selectfont]{subsubsection}

\usepackage{mwe}

% chapter style
\renewcommand*{\chapterformat}{%
  \thechapter\enskip
  \textcolor{gray!50}{\rule[-\dp\strutbox]{2pt}{\baselineskip}}\enskip
}
\setkomafont{disposition}{\normalcolor\bfseries}

%paragraph anpassen
%\addtokomafont{paragraph}{\itshape}
%\setkomafont{subsubsection}{\large}
%\setkomafont{paragraph}{\normalsize\itshape}
\setkomafont{paragraph}{\normalsize}

% layout of the paragraphs
% paragraphs look like the subsubsections
\RedeclareSectionCommands[
    beforeskip=-3.25ex plus -1ex minus -0.2ex,
    afterskip=1sp,% smallest possible positive value
]{paragraph,subparagraph}

% Roman page numbers in backmatter
\usepackage{etoolbox}
\makeatletter
\patchcmd{\backmatter}
{\@mainmatterfalse}
{\@mainmatterfalse\pagenumbering{roman}}
{}
{}
\makeatother


\usepackage{listings}
\usepackage[many]{tcolorbox}

% name of listings in the toc
\renewcommand\lstlistlistingname{Listingverzeichnis}

% general settings for listing
\lstset{
    xleftmargin=1.1cm,
    belowskip=2em,
    basicstyle=\fontsize{10}{15}\ttfamily,
    basewidth  = {.5em,0.4em},
    captionpos=t,
    lineskip={2pt},
    backgroundcolor=\color{white},
    framextopmargin=6pt,
    framexrightmargin=0pt,
    framexleftmargin=0.9em,    
    framexbottommargin=6pt, 
    frame=l,
%    frame=single,        
%    frame=tb, framerule=0pt,    
    framesep=6.5mm,
    fillcolor=\color{white},
    rulecolor=\color{middlegray},
    numbers=left,
    numberstyle=\normalfont\color{middlegray},
%    numberstyle=\footnotesize,    
    numbersep=10pt,
    abovecaptionskip=10pt, %space above the caption
    belowcaptionskip=10pt, %space below the caption
    extendedchars=true,
    showstringspaces=false,
    showspaces=false,
    stepnumber=1, % the step between two line-numbers. If it is 1 each line will be numbered
    tabsize=2,
    breaklines=true,
    showtabs=false,
    upquote=true,
    % German umlauts
    literate=%
    {Ö}{{\"O}}1
    {Ä}{{\"A}}1
    {Ü}{{\"U}}1
    {ß}{{\ss}}1
    {ü}{{\"u}}1
    {ä}{{\"a}}1
    {ö}{{\"o}}1
}

% define language
\lstdefinelanguage{JavaScript}{
    keywords={typeof, new, true, false, catch, then, function, return, null, catch, switch, var, if, in, while, do, else, case, break, default},
    keywordstyle=\color{editorGreen}\bfseries,
    ndkeywords={class, export, boolean, throw, implements, import, this, const},
    ndkeywordstyle=\color{vscodeblue}\bfseries,
    identifierstyle=\color{black},
    sensitive=false,
    comment=[l]{//},
    morecomment=[s]{/*}{*/},
    commentstyle=\color{editorGreen}\ttfamily,
    stringstyle=\color{darkred}\ttfamily,
    morestring=[b]',
    morestring=[b]"
}

\lstdefinelanguage{SQL}{
    keywords={select, where, from},
    keywordstyle=\color{editorGreen}\bfseries,
    ndkeywords={},
    ndkeywordstyle=\color{vscodeblue}\bfseries,
    identifierstyle=\color{black},
    sensitive=false,
    comment=[l]{--},
    morecomment=[s]{/*}{*/},
    commentstyle=\color{editorGreen}\ttfamily,
    stringstyle=\color{darkred}\ttfamily,
    morestring=[b]',
    morestring=[b]"
}


\lstdefinelanguage{HTML5}{
  language=html,
  sensitive=true,	
  alsoletter={<>=-},	
  morecomment=[s]{<!-}{-->},
  tag=[s],
  otherkeywords={
  % General
  >,
  % Standard tags
	<!DOCTYPE,
  </html, <html, <head, <title, </title, <style, </style, <link, </head, <meta, />,
	% body
	</body, <body,
	% Divs
	</div, <div, </div>, 
	% Paragraphs
	</p, <p, </p>,
	% scripts
	</script, <script,
  % More tags...
  <canvas, /canvas>, <svg, <rect, <animateTransform, </rect>, </svg>, <video, <source, <iframe, </iframe>, </video>, <image, </image>, <header, </header, <article, </article
  },
  ndkeywords={
  % General
  =,
  % HTML attributes
  charset=, src=, id=, width=, height=, style=, type=, rel=, href=,
  % SVG attributes
  fill=, attributeName=, begin=, dur=, from=, to=, poster=, controls=, x=, y=, repeatCount=, xlink:href=,
  % properties
  margin:, padding:, background-image:, border:, top:, left:, position:, width:, height:, margin-top:, margin-bottom:, font-size:, line-height:,
	% CSS3 properties
  transform:, -moz-transform:, -webkit-transform:,
  animation:, -webkit-animation:,
  transition:,  transition-duration:, transition-property:, transition-timing-function:,
  }
}

\lstdefinestyle{html} {%
  % Code design
  keywordstyle=\color{lightblack}\bfseries,
  ndkeywordstyle=\color{lightblack}\bfseries,
  identifierstyle=\color{lightblack},
  commentstyle=\color{green}\ttfamily,
  stringstyle=\color{darkred}\ttfamily,
  % Code
  language=HTML5,
%  alsolanguage=JavaScript,
  alsodigit={.:;},	
  tabsize=2,
  showtabs=false,
  showspaces=false,
  showstringspaces=false,
  extendedchars=true,
  breaklines=true,
  % German umlauts
  literate=%
  {Ö}{{\"O}}1
  {Ä}{{\"A}}1
  {Ü}{{\"U}}1
  {ß}{{\ss}}1
  {ü}{{\"u}}1
  {ä}{{\"a}}1
  {ö}{{\"o}}1
}

\lstdefinelanguage{CSS} 
{morekeywords={color,background,margin,padding,margin,padding,font,weight,display,position,top,left,right,bottom,list,style,border,size,white,space,min,width, 	transition}, 
	sensitive=false, 
	morecomment=[l]{//}, 
	morecomment=[s]{/*}{*/}, 
	morestring=[b]", 
}

\lstdefinestyle{css} {%
  language=CSS,
  keywordstyle=\color{lightblack},
}

\lstdefinestyle{js} {
  language=JavaScript
}

\lstdefinestyle{sql} {
  language=SQL,
  keywordstyle=\color{azure}  
}

%Usage
%\begin{minipage}{\linewidth}
%\begin{lstlisting}[style=js, caption={Flux Action Creator}, label=lst:actionCreator] 
%create: function(text) {
  %AppDispatcher.dispatch({
    %type: Constants.TODO_CREATE,
    %payload: 'sample'
  %});
%},
%\end{lstlisting}
%\end{minipage}
% for the links in toc
\usepackage{hyperref}
\hypersetup{
    colorlinks,
    citecolor=black,
    filecolor=black,
    linkcolor=black,
    urlcolor=black,
    pdfstartview= % fit zoom size to the viewer
}
%\newcommand{\code}[1]{\colorbox{lightgray}{\texttt{#1}}}
%\lstinline{snippet}
%http://tex.stackexchange.com/questions/65291/code-snippet-in-text

% This "\code{my code}" can be used to highlight small code snippets in the text like names of variables or methods.
\newcommand{\code}[1]{\textcolor{black}{\texttt{#1}}}

% The "\todo{this still has to be done}" is a command that highlights todos in the text.
\newcommand{\todo}[1]{\textcolor{vscodered}{TODO: \texttt{#1}}}

% Fußnoten
\usepackage{footnote}
\makesavenoteenv{figure}

\begin{document}

% PDF als Deckblatt einbinden
%\includepdf[pages=-,templatesize={145mm}{210mm},noautoscale=true,offset=-65 50]{myfile.pdf}
%\includepdf[pages={1},offset=25 -40]{chapters/deckblatt.pdf}

\titlehead{\centering University of Jon Doe}
\subject{Thesis}
\title{Title of the thesis}
\subtitle{If you want you can have a subtitle}
\author{Jon Doe}
\date{\today}
\publishers{Prof. Dr. rer. LaTeX}
\maketitle
\input{chapters/abstract}
\cleardoubleoddpage	
\chapter*{Eidesstattliche Erklärung}
\thispagestyle{empty} %hide page numbers

\enquote{Ich erkläre an Eides statt, dass ich die hier vorgelegte Bachelor-Thesis selbstständig und
ausschließlich unter Verwendung der angegebenen Literatur und sonstigen Hilfsmittel verfasst habe.
Die Arbeit wurde in gleicher oder ähnlicher Form keiner anderen Prüfungsbehörde zur Erlangung
eines akademischen Grades vorgelegt.}

\vspace{4cm}

\hspace{2cm} Ort, Datum \hfill Unterschrift \hspace{2cm}

\cleardoubleoddpage	
\newpage\thispagestyle{empty}
{
  \pagestyle{empty}
  \addtocontents{toc}{\protect\thispagestyle{empty}} 
  \tableofcontents
  \clearpage
}

\frontmatter
\pagenumbering{Roman} 

% Verzeichnisse
\listoffigures
\setcounter{page}{1}
\listoftables
\lstlistoflistings 


\mainmatter

% Kapitel einbinden
\chapter{Einleitung}
\setcounter{page}{1}
Lorem ipsum dolor sit amet \parencite[vgl.][215]{Bloggs2013}, consetetur sadipscing elitr, sed diam nonumy eirmod tempor invidunt ut labore et dolore magna aliquyam erat, sed diam voluptua. At vero eos et accusam et justo duo dolores et ea rebum. Stet clita kasd gubergren, no sea takimata sanctus est Lorem ipsum dolor sit amet. Lorem ipsum dolor sit amet, consetetur sadipscing elitr, sed diam nonumy eirmod tempor invidunt ut labore et dolore magna aliquyam erat, sed diam voluptua. At vero eos et accusam et justo duo dolores et ea rebum. Stet clita kasd gubergren, no sea takimata sanctus est Lorem ipsum dolor sit amet.

\section{Unterpunkt 1}
\textcite{Doe2015} beschreibt in seinem Ansatz\footcite{Doe2015}, dass sed diam nonumy eirmod tempor invidunt ut labore et dolore magna aliquyam erat \parencite[vgl.][26]{Doe2015}.
Lorem ipsum dolor sit amet, consetetur sadipscing elitr, sed diam nonumy eirmod tempor invidunt ut labore et dolore magna aliquyam erat, sed diam voluptua\footnote{Sed diam nonumy eirmod tempor invidunt ut labore et dolore magna aliquyam erat.}. At vero eos et accusam et justo duo dolores et ea rebum. Stet clita kasd gubergren, no sea takimata sanctus est Lorem ipsum dolor sit amet. Lorem ipsum dolor sit amet, consetetur sadipscing elitr, sed diam nonumy eirmod tempor invidunt ut labore et dolore magna aliquyam erat, sed diam voluptua. At vero eos et accusam et justo duo dolores et ea rebum. Stet clita kasd gubergren, no sea takimata sanctus est Lorem ipsum dolor sit amet.

\begin{figure}[h]
    \includegraphics[width=\textwidth, height=\textheight,keepaspectratio]{imageName}
    \caption[Beispielbild (Abbildungsverzeichnis)]{Beispielbild} 
    \label{fig:imageYouCanReferTo}
\end{figure}

In Abbildung \ref{fig:imageYouCanReferTo} kann man sehr gut erkennen, dass es sich um ein Beispielbild handelt. Lorem ipsum dolor sit amet, consetetur sadipscing elitr, sed diam nonumy eirmod tempor invidunt ut labore et dolore magna aliquyam erat, sed diam voluptua. At vero eos et accusam et justo duo dolores et ea rebum. Stet clita kasd gubergren, no sea takimata sanctus est Lorem ipsum dolor sit amet. Lorem ipsum dolor sit amet, consetetur sadipscing elitr, sed diam nonumy eirmod tempor invidunt ut labore et dolore magna aliquyam erat, sed diam voluptua. At vero eos et accusam et justo duo dolores et ea rebum. Stet clita kasd gubergren, no sea takimata sanctus est Lorem ipsum dolor sit amet.

Lorem ipsum dolor sit amet, consetetur sadipscing elitr, sed diam nonumy eirmod tempor invidunt ut labore et dolore magna aliquyam erat, sed diam voluptua. At vero eos et accusam et justo duo dolores et ea rebum. Stet clita kasd gubergren, no sea takimata sanctus est Lorem ipsum dolor sit amet. Lorem ipsum dolor sit amet, consetetur sadipscing elitr, sed diam nonumy eirmod tempor invidunt ut labore et dolore magna aliquyam erat, sed diam voluptua. At vero eos et accusam et justo duo dolores et ea rebum. Stet clita kasd gubergren, no sea takimata sanctus est Lorem ipsum dolor sit amet.

Lorem ipsum dolor sit amet, consetetur sadipscing elitr, sed diam nonumy eirmod tempor invidunt ut labore et dolore magna aliquyam erat, sed diam voluptua. At vero eos et accusam et justo duo dolores et ea rebum. Stet clita kasd gubergren, no sea takimata sanctus est Lorem ipsum dolor sit amet. Lorem ipsum dolor sit amet, consetetur sadipscing elitr, sed diam nonumy eirmod tempor invidunt ut labore et dolore magna aliquyam erat, sed diam voluptua. At vero eos et accusam et justo duo dolores et ea rebum. Stet clita kasd gubergren, no sea takimata sanctus est Lorem ipsum dolor sit amet.

\bgroup
\def\arraystretch{2}
\begin{table}[h]
\centering
\caption{Beispieltabelle}
\label{tbl:libs}
\begin{tabular}{|l|l@{\hspace{2em}}|l@{\hspace{2em}}|}
\hline
\diagbox{Punkt 1}{Punkt 2}  & Beispiel & Beispiel     \\ \hline
Beispiel & Beispiel & Beispiel \\ \hline
Beispiel     & Beispiel  & Beispiel        \\ \hline
\end{tabular}
\end{table}
\egroup

\section{Unterpunkt 2}
Lorem ipsum dolor sit amet, consetetur sadipscing elitr, sed diam nonumy eirmod tempor invidunt ut labore et dolore magna aliquyam erat, sed diam voluptua. At vero eos et accusam et justo duo dolores et ea rebum. Stet clita kasd gubergren, no sea takimata sanctus est Lorem ipsum dolor sit amet. Lorem ipsum dolor sit amet, consetetur sadipscing elitr, sed diam nonumy eirmod tempor invidunt ut labore et dolore magna aliquyam erat, sed diam voluptua. At vero eos et accusam et justo duo dolores et ea rebum. Stet clita kasd gubergren, no sea takimata sanctus est Lorem ipsum dolor sit amet.

\begin{figure}[h]
    \centering
    \includegraphics[width=300px, height=\textheight,keepaspectratio]{imageName}
    \caption[weiteres Beispielbild]{Beispielbild}
    \label{fig:sampleImage}
\end{figure}

Lorem ipsum dolor sit amet, consetetur sadipscing elitr, sed diam nonumy eirmod tempor invidunt ut labore et dolore magna aliquyam erat, sed diam voluptua. At vero eos et accusam et justo duo dolores et ea rebum. Stet clita kasd gubergren, no sea takimata sanctus est Lorem ipsum dolor sit amet. Lorem ipsum dolor sit amet, consetetur sadipscing elitr, sed diam nonumy eirmod tempor invidunt ut labore et dolore magna aliquyam erat, sed diam voluptua. At vero eos et accusam et justo duo dolores et ea rebum. Stet clita kasd gubergren, no sea takimata sanctus est Lorem ipsum dolor sit amet.

Lorem ipsum dolor sit amet, consetetur sadipscing elitr, sed diam nonumy eirmod tempor invidunt ut labore et dolore magna aliquyam erat, sed diam voluptua. At vero eos et accusam et justo duo dolores et ea rebum. Stet clita kasd gubergren, no sea takimata sanctus est Lorem ipsum dolor sit amet. Lorem ipsum dolor sit amet, consetetur sadipscing elitr, sed diam nonumy eirmod tempor invidunt ut labore et dolore magna aliquyam erat, sed diam voluptua. At vero eos et accusam et justo duo dolores et ea rebum. Stet clita kasd gubergren, no sea takimata sanctus est Lorem ipsum dolor sit amet.

\section{Unterpunkt 3}
\textcolor{editorOcher}{Lorem ipsum dolor} sit amet, consetetur sadipscing elitr, sed diam nonumy eirmod tempor invidunt ut labore et dolore magna aliquyam erat, sed diam voluptua. At vero eos et accusam et justo duo dolores et ea rebum. Stet clita kasd gubergren, no sea takimata sanctus est Lorem ipsum dolor sit amet. Lorem ipsum dolor sit amet, consetetur sadipscing elitr, sed diam nonumy eirmod tempor invidunt ut labore et dolore magna aliquyam erat, sed diam voluptua. At vero eos et accusam et justo duo dolores et ea rebum. Stet clita kasd gubergren, no sea takimata sanctus est Lorem ipsum dolor sit amet.

Lorem ipsum dolor sit amet, consetetur sadipscing elitr, sed diam nonumy eirmod tempor invidunt ut labore et dolore magna aliquyam erat, sed diam voluptua. At vero eos et accusam et justo duo dolores et ea rebum. Stet clita kasd gubergren, no sea takimata sanctus est Lorem ipsum dolor sit amet. Lorem ipsum dolor sit amet, consetetur sadipscing elitr, sed diam nonumy eirmod tempor invidunt ut labore et dolore magna aliquyam erat, sed diam voluptua. At vero eos et accusam et justo duo dolores et ea rebum. Stet clita kasd gubergren, no sea takimata sanctus est Lorem ipsum dolor sit amet.
\chapter{Fragestellung}
Lorem ipsum dolor sit amet, consetetur sadipscing elitr, sed diam nonumy eirmod tempor invidunt ut labore et dolore magna aliquyam erat, sed diam voluptua. At vero eos et accusam et justo duo dolores et ea rebum. Stet clita kasd gubergren, no sea takimata sanctus est Lorem ipsum dolor sit amet. Lorem ipsum dolor sit amet, consetetur sadipscing elitr, sed diam nonumy eirmod tempor invidunt ut labore et dolore magna aliquyam erat, sed diam voluptua. At vero eos et accusam et justo duo dolores et ea rebum. Stet clita kasd gubergren, no sea takimata sanctus est Lorem ipsum dolor sit amet.

\section{Unterabschnitt}
Lorem ipsum dolor sit amet, consetetur sadipscing elitr, sed diam nonumy eirmod tempor invidunt ut labore et dolore magna aliquyam erat, sed diam voluptua. At vero eos et accusam et justo duo dolores et ea rebum. Stet clita kasd gubergren, no sea takimata sanctus est Lorem ipsum dolor sit amet. Lorem ipsum dolor sit amet, consetetur sadipscing elitr, sed diam nonumy eirmod tempor invidunt ut labore et dolore magna aliquyam erat, sed diam voluptua. At vero eos et accusam et justo duo dolores et ea rebum. Stet clita kasd gubergren, no sea takimata sanctus est Lorem ipsum dolor sit amet.

Lorem ipsum dolor sit amet, consetetur sadipscing elitr, sed diam nonumy eirmod tempor invidunt ut labore et dolore magna aliquyam erat, sed diam voluptua. At vero eos et accusam et justo duo dolores et ea rebum. Stet clita kasd gubergren, no sea takimata sanctus est Lorem ipsum dolor sit amet. Lorem ipsum dolor sit amet, consetetur sadipscing elitr, sed diam nonumy eirmod tempor invidunt ut labore et dolore magna aliquyam erat, sed diam voluptua. At vero eos et accusam et justo duo dolores et ea rebum. Stet clita kasd gubergren, no sea takimata sanctus est Lorem ipsum dolor sit amet.

Lorem ipsum dolor sit amet, consetetur sadipscing elitr, sed diam nonumy eirmod tempor invidunt ut labore et dolore magna aliquyam erat, sed diam voluptua. At vero eos et accusam et justo duo dolores et ea rebum. Stet clita kasd gubergren, no sea takimata sanctus est Lorem ipsum dolor sit amet. Lorem ipsum dolor sit amet, consetetur sadipscing elitr, sed diam nonumy eirmod tempor invidunt ut labore et dolore magna aliquyam erat, sed diam voluptua. At vero eos et accusam et justo duo dolores et ea rebum. Stet clita kasd gubergren, no sea takimata sanctus est Lorem ipsum dolor sit amet.

Lorem ipsum dolor sit amet, consetetur sadipscing elitr, sed diam nonumy eirmod tempor invidunt ut labore et dolore magna aliquyam erat, sed diam voluptua. At vero eos et accusam et justo duo dolores et ea rebum. Stet clita kasd gubergren, no sea takimata sanctus est Lorem ipsum dolor sit amet. Lorem ipsum dolor sit amet, consetetur sadipscing elitr, sed diam nonumy eirmod tempor invidunt ut labore et dolore magna aliquyam erat, sed diam voluptua. At vero eos et accusam et justo duo dolores et ea rebum. Stet clita kasd gubergren, no sea takimata sanctus est Lorem ipsum dolor sit amet.

Lorem ipsum dolor sit amet, consetetur sadipscing elitr, sed diam nonumy eirmod tempor invidunt ut labore et dolore magna aliquyam erat, sed diam voluptua. At vero eos et accusam et justo duo dolores et ea rebum. Stet clita kasd gubergren, no sea takimata sanctus est Lorem ipsum dolor sit amet. Lorem ipsum dolor sit amet, consetetur sadipscing elitr, sed diam nonumy eirmod tempor invidunt ut labore et dolore magna aliquyam erat, sed diam voluptua. At vero eos et accusam et justo duo dolores et ea rebum. Stet clita kasd gubergren, no sea takimata sanctus est Lorem ipsum dolor sit amet.

\section{Unterabschnitt}
Lorem ipsum dolor sit amet, consetetur sadipscing elitr, sed diam nonumy eirmod tempor invidunt ut labore et dolore magna aliquyam erat, sed diam voluptua. At vero eos et accusam et justo duo dolores et ea rebum. Stet clita kasd gubergren, no sea takimata sanctus est Lorem ipsum dolor sit amet. Lorem ipsum dolor sit amet, consetetur sadipscing elitr, sed diam nonumy eirmod tempor invidunt ut labore et dolore magna aliquyam erat, sed diam voluptua. At vero eos et accusam et justo duo dolores et ea rebum. Stet clita kasd gubergren, no sea takimata sanctus est Lorem ipsum dolor sit amet.

\section{Unterabschnitt}
Lorem ipsum dolor sit amet, consetetur sadipscing elitr, sed diam nonumy eirmod tempor invidunt ut labore et dolore magna aliquyam erat, sed diam voluptua. At vero eos et accusam et justo duo dolores et ea rebum. Stet clita kasd gubergren, no sea takimata sanctus est Lorem ipsum dolor sit amet. Lorem ipsum dolor sit amet, consetetur sadipscing elitr, sed diam nonumy eirmod tempor invidunt ut labore et dolore magna aliquyam erat, sed diam voluptua. At vero eos et accusam et justo duo dolores et ea rebum. Stet clita kasd gubergren, no sea takimata sanctus est Lorem ipsum dolor sit amet.

\subsection{Unterabschnitt}
Lorem ipsum dolor sit amet, consetetur sadipscing elitr, sed diam nonumy eirmod tempor invidunt ut labore et dolore magna aliquyam erat, sed diam voluptua. At vero eos et accusam et justo duo dolores et ea rebum. Stet clita kasd gubergren, no sea takimata sanctus est Lorem ipsum dolor sit amet. Lorem ipsum dolor sit amet, consetetur sadipscing elitr, sed diam nonumy eirmod tempor invidunt ut labore et dolore magna aliquyam erat, sed diam voluptua. At vero eos et accusam et justo duo dolores et ea rebum. Stet clita kasd gubergren, no sea takimata sanctus est Lorem ipsum dolor sit amet.

\paragraph{Unterabschnitt}
Lorem ipsum dolor sit amet, consetetur sadipscing elitr, sed diam nonumy eirmod tempor invidunt ut labore et dolore magna aliquyam erat, sed diam voluptua. At vero eos et accusam et justo duo dolores et ea rebum. Stet clita kasd gubergren, no sea takimata sanctus est Lorem ipsum dolor sit amet. Lorem ipsum dolor sit amet, consetetur sadipscing elitr, sed diam nonumy eirmod tempor invidunt ut labore et dolore magna aliquyam erat, sed diam voluptua. At vero eos et accusam et justo duo dolores et ea rebum. Stet clita kasd gubergren, no sea takimata sanctus est Lorem ipsum dolor sit amet.

\paragraph{Unterabschnitt}
Lorem ipsum dolor sit amet, consetetur sadipscing elitr, sed diam nonumy eirmod tempor invidunt ut labore et dolore magna aliquyam erat, sed diam voluptua. At vero eos et accusam et justo duo dolores et ea rebum. Stet clita kasd gubergren, no sea takimata sanctus est Lorem ipsum dolor sit amet. Lorem ipsum dolor sit amet, consetetur sadipscing elitr, sed diam nonumy eirmod tempor invidunt ut labore et dolore magna aliquyam erat, sed diam voluptua. At vero eos et accusam et justo duo dolores et ea rebum. Stet clita kasd gubergren, no sea takimata sanctus est Lorem ipsum dolor sit amet.

\subsection{Unterabschnitt}
Lorem ipsum dolor sit amet, consetetur sadipscing elitr, sed diam nonumy eirmod tempor invidunt ut labore et dolore magna aliquyam erat, sed diam voluptua. At vero eos et accusam et justo duo dolores et ea rebum. Stet clita kasd gubergren, no sea takimata sanctus est Lorem ipsum dolor sit amet. Lorem ipsum dolor sit amet, consetetur sadipscing elitr, sed diam nonumy eirmod tempor invidunt ut labore et dolore magna aliquyam erat, sed diam voluptua. At vero eos et accusam et justo duo dolores et ea rebum. Stet clita kasd gubergren, no sea takimata sanctus est Lorem ipsum dolor sit amet.

\begin{minipage}{\linewidth}
\begin{lstlisting}[style=js, caption={Beispiellisting}, label=lst:sample] 
function hello(world){
    console.log('hello ' + world);
}
\end{lstlisting}
\end{minipage}

\autoref{lst:sample} zeigt ein beispielhaftes Listing. Der Code \code{console.log} sorgt dafür, dass etwas auf der Konsole ausgegeben wird.

Lorem ipsum dolor sit amet, consetetur sadipscing elitr, sed diam nonumy eirmod tempor invidunt ut labore et dolore magna aliquyam erat, sed diam voluptua. At vero eos et accusam et justo duo dolores et ea rebum. Stet clita kasd gubergren, no sea takimata sanctus est Lorem ipsum dolor sit amet. Lorem ipsum dolor sit amet, consetetur sadipscing elitr, sed diam nonumy eirmod tempor invidunt ut labore et dolore magna aliquyam erat, sed diam voluptua. At vero eos et accusam et justo duo dolores et ea rebum. Stet clita kasd gubergren, no sea takimata sanctus est Lorem ipsum dolor sit amet.
\chapter{Hauptteil}
\label{cha:hauptteil}
Lorem ipsum dolor sit amet, consetetur sadipscing elitr, sed diam nonumy eirmod tempor invidunt ut labore et dolore magna aliquyam erat, sed diam voluptua. At vero eos et accusam et justo duo dolores et ea rebum. Stet clita kasd gubergren, no sea takimata sanctus est Lorem ipsum dolor sit amet. Lorem ipsum dolor sit amet, consetetur sadipscing elitr, sed diam nonumy eirmod tempor invidunt ut labore et dolore magna aliquyam erat, sed diam voluptua. At vero eos et accusam et justo duo dolores et ea rebum. Stet clita kasd gubergren, no sea takimata sanctus est Lorem ipsum dolor sit amet.


\section{Unterabschnitt}
In \autoref{cha:hauptteil} wurde bereits beschrieben...
Lorem ipsum dolor sit amet, consetetur sadipscing elitr, sed diam nonumy eirmod tempor invidunt ut labore et dolore magna aliquyam erat, sed diam voluptua. At vero eos et accusam et justo duo dolores et ea rebum. Stet clita kasd gubergren, no sea takimata sanctus est Lorem ipsum dolor sit amet. Lorem ipsum dolor sit amet, consetetur sadipscing elitr, sed diam nonumy eirmod tempor invidunt ut labore et dolore magna aliquyam erat, sed diam voluptua. At vero eos et accusam et justo duo dolores et ea rebum. Stet clita kasd gubergren, no sea takimata sanctus est Lorem ipsum dolor sit amet.

Lorem ipsum dolor sit amet, consetetur sadipscing elitr, sed diam nonumy eirmod tempor invidunt ut labore et dolore magna aliquyam erat, sed diam voluptua. At vero eos et accusam et justo duo dolores et ea rebum. Stet clita kasd gubergren, no sea takimata sanctus est Lorem ipsum dolor sit amet. Lorem ipsum dolor sit amet, consetetur sadipscing elitr, sed diam nonumy eirmod tempor invidunt ut labore et dolore magna aliquyam erat, sed diam voluptua. At vero eos et accusam et justo duo dolores et ea rebum. Stet clita kasd gubergren, no sea takimata sanctus est Lorem ipsum dolor sit amet.

\section{Unterabschnitt}
In \autoref{cha:hauptteil} wurde bereits beschrieben...
Lorem ipsum dolor sit amet, consetetur sadipscing elitr, sed diam nonumy eirmod tempor invidunt ut labore et dolore magna aliquyam erat, sed diam voluptua. At vero eos et accusam et justo duo dolores et ea rebum. Stet clita kasd gubergren, no sea takimata sanctus est Lorem ipsum dolor sit amet. Lorem ipsum dolor sit amet, consetetur sadipscing elitr, sed diam nonumy eirmod tempor invidunt ut labore et dolore magna aliquyam erat, sed diam voluptua. At vero eos et accusam et justo duo dolores et ea rebum. Stet clita kasd gubergren, no sea takimata sanctus est Lorem ipsum dolor sit amet.

Lorem ipsum dolor sit amet, consetetur sadipscing elitr, sed diam nonumy eirmod tempor invidunt ut labore et dolore magna aliquyam erat, sed diam voluptua. At vero eos et accusam et justo duo dolores et ea rebum. Stet clita kasd gubergren, no sea takimata sanctus est Lorem ipsum dolor sit amet. Lorem ipsum dolor sit amet, consetetur sadipscing elitr, sed diam nonumy eirmod tempor invidunt ut labore et dolore magna aliquyam erat, sed diam voluptua. At vero eos et accusam et justo duo dolores et ea rebum. Stet clita kasd gubergren, no sea takimata sanctus est Lorem ipsum dolor sit amet.
\input{chapters/chapter4}
\input{chapters/chapter5}

\backmatter
\pagenumbering{Roman} 
\setcounter{page}{6} % römisch weiterzählen (von frontmatter ab)

% Anhang
\appendix
%\chapter{Anhang} - kein A vorne
\addchap{Anhang}
\pagenumbering{Roman} 
\setcounter{page}{4}

\subsection*{Anhang 1}
Anhang 1

\subsection*{Anhang 2}
Anhang 1


% Literaturverzeichnis
\cleardoublepage
\printbibheading[title=Quellen]
\printbibliography[heading=subbibliography, type=online, title={Online}]
\printbibliography[heading=subbibliography, type=book, title={Literatur}]
%\printbibliography[heading=subbibliography,title={Der ganze Rest},nottype=online,notkeyword=Norm,notkeyword=sekundaer]

\end{document}


